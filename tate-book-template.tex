\documentclass[ % ドキュメントクラス
    uplatex, % upLaTeXを使う
    tate, % 縦書き
%    twocolumn, % 二段組
    hanging_punctuation, % ぶら下げ組
    paper=b6, % 用紙サイズ
    reference_mark=interlinear, %注の合標を行間に
    book % 書籍のためのスタイル
]{jlreq}

%% フォント関連
\usepackage[T1]{fontenc} % フォントでT1を使うこと
\usepackage{textcomp} % フォントでTS1を使うこと
\usepackage[utf8]{inputenc} % ファイルがUTF8であること
\usepackage[multi,deluxe,jis2004]{otf}
\usepackage[prefernoncjk]{pxcjkcat} % なるべく「半角」扱いで。

%% 図表など
% 図の読みこみのために
\usepackage[dvipdfmx, hiresbb]{graphicx, xcolor}
\usepackage{booktabs} % 表の横罫線

%% 囲み枠
\usepackage{tcolorbox}
\tcbuselibrary{breakable} % ページをまたいで分割できるように

%% misc
\usepackage{okumacro} % 圏点などのために
\usepackage{pxrubrica} % ルビをつける(okumacroのrubyは使わない)
\usepackage{sfkanbun} % 漢文
% sfkanbun パッケージは、
% http://xymtex.my.coocan.jp/fujitas2/texlatex/index.html 
% から入手可能

%% 見出しのスタイルの設定
% chapterの定義
\DeclareTobiraHeading{chapter}{1}{% chapter を扉見出しに
  format={\null\vfil {\huge\gtfamily\bfseries {\LARGE #1}#2}}, % 見出しのフォント
  label_format={第\thechapter 章\hspace{2zw}} % ラベルのフォーマット
}
\DeclareBlockHeading{chapter}{1}{ % chapter を別行見出しに
  pagebreak=cleardoublepage, % 章を始める前に改丁
  label_format={第\thechapter 章}, % ラベルのフォーマット
  font={\gtfamily\LARGE}, % 見出しのフォント
  lines=3,after_lines=2, % 見出しのために5行取り、後ろの方が2行分広い
  indent=2zw % インデント
}

% section の定義
\renewcommand{\thesection}{} % 節の番号はなしが基本
\DeclareBlockHeading{section}{2}{ % section を別行見出しに
  font={\gtfamily}, % 見出しのフォント
  lines=1, before_lines=1% 見出しの前に1行取る
}

%% 目次の設定
\setcounter{tocdepth}{1} % sectionまでを目次に

%% hyperrefの設定
\usepackage[dvipdfmx,%
    pdftitle=タイトル, % PDFのタイトル
    pdfauthor=作成者, % PDFの作成者
    bookmarks=true, % PDFにしおりをつける
    bookmarksnumbered=true, % しおりに節番号などをつける
    colorlinks=false, % リンクには色をつけない
    hyperfootnotes=false, % 脚注からのリンクを作らない
    pdfborder={0 0 0}, % 枠なし
    pdfdirection=R2L, % 開く方向
    pdfpagelayout=TwoPageRight, % 奇数頁が右側になるような見開きモードで開く
    pdfpagemode=UseNone]{hyperref}

% PDFにしたときのしおりの文字化けを防ぐ
\usepackage{pxjahyper}

% hyperref を使っているときに
% 目次でのページ番号の向きを適切にする
\makeatletter
\def\contentsline#1#2#3#4{\csname l@#1\endcsname{\hyper@linkstart{link}{#4}{#2}\hyper@linkend}{\rensuji{#3}}}
\makeatother


\begin{document}
% maketitle を使わずに独自のタイトルページを作る
\begin{titlepage}
  \vspace*{10mm}
  \noindent{\fontsize{30pt}{48pt}\gtfamily\bfseries タイトル}
  \vfill

  \begin{flushright}
    {\gtfamily\bfseries\huge 著者 名}
  \end{flushright}
\end{titlepage}

\phantomsection
\addcontentsline{toc}{chapter}{序}
\chapter*{序}

ここには序の内容が入る。

\tableofcontents % 目次


\chapter{最初の章}

ここは最初の章の冒頭の文章が入る。
ここは最初の章の冒頭の文章が入る。


\section{最初の節の見出し}

ここは最初の節の文章が入る。
ここは最初の節の文章が入る。
ここは最初の節の文章が入る。
ここは最初の節の文章が入る。
ここは最初の節の文章が入る。


\section{第二の節の見出し}

ここは第二の節の文章が入る。
ここは第二の節の文章が入る。
ここは第二の節の文章が入る。
ここは第二の節の文章が入る。
ここは第二の節の文章が入る。


\chapter{便利な命令}

\section{文字装飾}

  \bou{傍点}・\kenten{圏点}・\kasen{傍線}

\section{特殊文字}

  \ajMaru{0} \ajMaru{5} \ajMaru{42} \ajMaru{100}
  \ajMaru*{0} \ajMaru*{5} \ajMaru*{42} \ajMaru*{100}
  \ajKuroMaru{0} \ajKuroMaru{5} \ajKuroMaru{42} \ajKuroMaru{100}
  \ajKuroMaru*{0} \ajKuroMaru*{5} \ajKuroMaru*{42} \ajKuroMaru*{100}

  \ajKaku{0} \ajKaku{5} \ajKaku{42} \ajKaku{100}
  \ajKaku*{0} \ajKaku*{5} \ajKaku*{42} \ajKaku*{100}
  \ajKuroKaku{0} \ajKuroKaku{5} \ajKuroKaku{42} \ajKuroKaku{100}
  \ajKuroKaku*{0} \ajKuroKaku*{5} \ajKuroKaku*{42} \ajKuroKaku*{100}

  \ajMaruKaku{0} \ajMaruKaku{5} \ajMaruKaku{42} \ajMaruKaku{100}
  \ajMaruKaku*{0} \ajMaruKaku*{5} \ajMaruKaku*{42} \ajMaruKaku*{100}
  \ajKuroMaruKaku{0} \ajKuroMaruKaku{5} \ajKuroMaruKaku{42} \ajKuroMaruKaku{100}
  \ajKuroMaruKaku*{0} \ajKuroMaruKaku*{5} \ajKuroMaruKaku*{42} \ajKuroMaruKaku*{100}


「こら〳〵」「どれ〴〵」

参加者は\,\rensuji{12}\,人だった。


\section{注釈}

脚注\sidenote{脚注。}を表示する。

後注\endnote{これが後注の文章である。}を表示する。

割注\warichu{これが割注の文章である。}を表示する。


\section{漢文}

% \kundoku{漢字}{ルビ}{送り仮名}{返り点}[肩返り点]<左送り仮名>(句読点)

\kundoku{未}{いま}{ダ}{レ}<ル>
\kundoku{知}{}{ラ}{二}
仁
\kundoku{義}{}{ヲ}{一}
\kundoku{也}{}{}{}(。)

\end{document}